This section is devoted to a specific description of every kind of requirement the system has to deal with in order to achieve all the functionalities described.

\newpage

\section{External interface requirements}
\label{sec:external_interface_requirements}%

\subsection{User interfaces}
\label{subsec:user_interfaces}%
The CodeKataBattle’s user interface will be a website, developed in order to be used both by STs and EDs, available to everybody who has a device with an Internet Browser and a reliable Internet connection.

\subsection{Hardware interfaces}
\label{subsec:hardware_interfaces}%
The system will be accessible from every device with an Internet Browser to access the website and a reliable Internet connection.
The user is free to choose his device like a computer, a tablet or a smartphone, despite that, it is suggested to use a computer, which will make it easier to deal with the writing of the code.


\subsection{Software interfaces}
\label{subsec:software_interfaces}%
The system requires the GitHub APIs and the External Tools in order to work properly. Also, it will use a mailing system to send confirmation emails to the users during the registration process.    

\subsection{Communication interfaces}
\label{subsec:communication_interfaces}%
The communication Interfaces needed by the system are the HTTPS Protocol and the Mail System Transfer Protocol (SMTP).

\section{Functional requirements}
\label{sec:functional_requirements}%

\subsection{Requirements}
\label{subsec:requirements3}%
\newcounter{req}
\setcounter{req}{1}
\newcommand{\creq}{\thereq\stepcounter{req}}
\begin{center}
    \begin{longtable}{|l|p{0.9\linewidth}|}
        \hline
        \textbf{ID} & \textbf{Description}                                                                                                                             \\
        \hline
        R\creq      & CKB allows unregistered Users to sign up                                                                    \\
        \hline
        R\creq      & CKB allows registered EDs to login                                                                    \\
        \hline
        R\creq      & CKB allows registered STs to login                                                                    \\
        \hline
        R\creq      & CKB allows EDs to create Tournaments                                                                    \\
        \hline
        R\creq      & CKB allows EDs to grant the permissions of a Tournament to other EDs                                                                 \\
        \hline
        R\creq      & CKB allows EDs to create Battles                                                                    \\
        \hline
        R\creq      & CKB allows EDs to uploads the code kata of a Battle                                                                   \\
        \hline
        R\creq      & CKB allows EDs to set the minimun and the maximum number of STs per group of a Battle                                                                    \\
        \hline
        R\creq      & CKB allows EDs to set a registration deadline of a Battle                                                                 \\
        \hline
        R\creq      & CKB allows EDs to set a submission deadline of a Battle                                                                \\
        \hline
        R\creq      & CKB allows EDs to set additional configuration for the scoring system of a Battle                                                                \\
        \hline
        R\creq      & CKB allows EDs to set functional aspects for the scoring system of a Battle                                                               \\
        \hline
        R\creq      & CKB allows EDs to create new badges                                                               \\
        \hline
        R\creq      & CKB allows EDs to choose the rules related to the awarding of badges                                                               \\
        \hline
        R\creq      & CKB allows EDs to choose which badges to award in a certain Tournament                                                               \\
        \hline
        R\creq      & CKB allows EDs to assing a score manually during the consolidation stage                                                               \\
        \hline
        R\creq      & CKB allows EDs to close a Tournament                                                               \\
        \hline
        R\creq      & CKB allows EDs to visualize the profile of another User                                                               \\
        \hline
        R\creq      & CKB allows STs to visualize the profile of another User \\
        \hline
        R\creq      & CKB allows STs to join a Tournament \\
        \hline
        R\creq      & CKB allows STs to join a Battle \\
        \hline
        R\creq      & CKB allows STs to create a new STG \\
        \hline
        R\creq      & CKB allows STs to join a STG \\
        \hline
        R\creq      & CKB allows STs to invite other STs in their STG \\
        \hline
        R\creq      & CKB stores the informations about the Users \\
        \hline
        R\creq      & CKB shall ensure security of data \\
        \hline
        R\creq      & CKB sends notifications to every ST when a new Tournament is created \\
        \hline
        R\creq      & CKB sends notifications when a new Battle is created to every ST which is participating in the Tournament that the Battle is part of \\
        \hline
        R\creq      & CKB sends notifications to a ST when he receives an invitation to be part of STG \\
        \hline
        R\creq      & CKB creates a GH repository of the code kata when the registration deadline for the Battle expires \\
        \hline
        R\creq      & CKB sends the link of the GH repository to every STG that participates in the Battle \\
        \hline
        R\creq      & CKB evaluates the STG's work every time a push is made on GH and calculates Battle score for the STG \\
        \hline
        R\creq      & CKB updates the Battle leaderboard once a new score is registered \\
        \hline
        R\creq      & CKB updates the Tournament leaderboard once a new score is registered \\
        \hline
        R\creq      & CKB allows STs to check the leaderboard of a Battle \\
        \hline
        R\creq      & CKB allows EDs to check the leaderboard of a Battle \\
        \hline
        R\creq      & CKB allows EDs to analyze the code of a STG \\
        \hline
        R\creq      & CKB sends notifications to every STs participating in the Battle once the consolidation stage ends \\
        \hline
        R\creq      & CKB allows STs to check the list of ongoing Tournaments \\
        \hline
        R\creq      & CKB allows EDs to check the list of ongoing Tournaments \\
        \hline
        R\creq      & CKB allows STs to check the leaderboard of a Tournaments \\
        \hline
        R\creq      & CKB allows EDs to check the leaderboard of a Tournaments \\
        \hline
        R\creq      & CKB sends notification to every ST involved in a Tournament when the Tournament is closed and the final ranks are available \\
        \hline
        R\creq      & CKB shall communicate with the GH API in order to calculate a new score every time a push action is made by a STG \\
        \hline
        R\creq      & CKB shall communicate with the external tool in order to calculate the score of a STG \\
        \hline
        R\creq      & CKB shall communicate with the mailing system in order to allow Users to register their account\\
        \hline
        R\creq      & STs need to fork the GH repository of the Battle they are participating in \\
        \hline
        R\creq      & STs need to push their code in the GH repository in order to have their code evaluated\\
        \hline
        R\creq      & CKB shall assign the badges to all STs that fullfill their requirements \\
        \hline
        \caption{Requirements}
        \label{tab: req}%
    \end{longtable}
\end{center}


\subsection{Mapping on goals}
\label{subsec:mapping_on_goals}%
bbbbbbbbbbbbbbbb

\subsection{Use case diagrams}
\label{subsec:use_case_diagrams}%
\subsection{Use cases}
\label{subsec: use_cases}%
\newcounter{uc}
\setcounter{uc}{1}
\newcommand{\cuc}{\theuc\stepcounter{uc}}
%sistemare
In this section, they are explained and represented the main identified use cases.
There is a table with entry conditions, event flow, exit conditions and exception for each of them, and a sequence diagram
that shows the messages exchanged between the entities and the called functions. \\


\subsubsection*{UC\cuc . Signup as ED}
\begin{center}
    \begin{longtable}{|l|p{0.9\linewidth}|}
        \hline
        \textbf{Actor}            & ED, email provider                                                                                                                                                                                        \\
        \hline
        \textbf{Entry conditions} & The ED is not already registered in CKB and has to search the CKB URL in the browser search bar                                                                                                                   \\
        \hline
        \textbf{Event Flow}       & 1 - The application shows the login form  \\
        & 2 - The ED clicks on “create an account” button   \\
        & 3 - The application shows the signup form    \\
        & 4 - The ED inserts his name, surname, nickname, email and password in the form and also ticks on the “Signup as Educator” checkbox  \\
        & 5 - The ED clicks on the “Register” button  \\
        & 6 - The application checks all the credentials  \\
        & 7 - If credentials are correct the application sends a confirmation email to the ED through the email provider.  \\
        & 8 - The ED clicks on the confirmation link                                                                         \\                                                                                                                                                            \\
        \hline
        \textbf{Exit condition}   & The application allows the ED to access to the CKB system \\                                                                                                                                                                                \\
        \hline
        \textbf{Exceptions}       & \begin{itemize}
            \item The email address is already linked to an account. In this case an error message is shown and the ED is redirected to the profile creation settings.
            \item Invalid password if it is shorter than 8 characters, if it doesn’t have at least 1 number and/or 1 capital letter and/or a special character. In this case an error message is shown and the ED is redirected to the profile creation settings.
            \item  The nickname is already used. An error is shown and the ED is redirected to the profile creation settings.
        \end{itemize}  \\
        \hline
        \caption{Signup as ED use case}
        \label{tab: signup_as_ED_use_case}
    \end{longtable}
\end{center}

\subsubsection*{UC\cuc . Signup as ST}
\begin{center}
    \begin{longtable}{|l|p{0.9\linewidth}|}
        \hline
        \textbf{Actor}            & ST, email provider                                                                                                                                                                                      \\
        \hline
        \textbf{Entry conditions} & The ST is not already registered in CKB and has to search the CKB URL in the browser search bar                                                                                                                   \\
        \hline
        \textbf{Event Flow}       & 1 - The application shows the login form  \\
        & 2 - The ST clicks on “create an account” button  \\
        & 3 - The application shows the signup form  \\
        & 4 - the ST inserts his name,surname, nickname and password in the form  \\
        & 5 - The ST clicks on the “Register” button  \\
        & 6 - The application checks all the credentials  \\
        & 7 - if credentials are correct the application sends a confirmation email to the ST through an email provider.  \\
        & 8 - The ST clicks on the confirmation link                                                                        \\                                                                                                                                                                    \\
        \hline
        \textbf{Exit condition}   & The application allows the ST to access to the CKB system \\                                                                                                                                                                                \\
        \hline
        \textbf{Exceptions}       & \begin{itemize}
            \item The email address is already linked to an account. In this case an error message is shown and the ST is redirected to the profile creation settings.
            \item Invalid password if it is shorter than 8 characters, if it doesn’t have at least 1 number and/or 1 capital letter and/or a special character. In this case an error message is shown and the ST is redirected to the profile creation settings.
            \item The nickname is already used. An error is shown and the ST is redirected to the profile creation settings.
        \end{itemize}    \\
        \hline
        \caption{Signup as ST use case}
        \label{tab: signup_as_ST_use_case}
    \end{longtable}
\end{center}

\subsubsection*{UC\cuc . Login}
\begin{center}
    \begin{longtable}{|l|p{0.9\linewidth}|}
        \hline
        \textbf{Actor}            & Users                                                                                                                                                                                        \\
        \hline
        \textbf{Entry conditions} & The User should be registered in CKB and has to search the CKB URL in the browser search bar                                                                                                                   \\
        \hline
        \textbf{Event Flow}       & 1-\ The application shows the login form                                                     \\
        & 2-\ The User insert his mail and password in the form \\
        & 3-\ The User clicks on the "Login" button       \\                                                                                                                      
        & 4-\ The system check the credentials  \\                                                                                                                                                                                                                                     \\
        \hline
        \textbf{Exit condition}   & The application allows the user to access to the CKB system \\                                                                                                                                                                                \\
        \hline
        \textbf{Exceptions}       & Incorrect email or password. An error message is shown and the User is redirected back to the Login page                                                                                                                        \\
        \hline
        \caption{Login use case}
        \label{tab: login_use_case}
    \end{longtable}
\end{center}

\subsubsection*{UC\cuc . Create a Tournament}
\begin{center}
    \begin{longtable}{|l|p{0.9\linewidth}|}
        \hline
        \textbf{Actor}            & ED                                                                                                                                                                                       \\
        \hline
        \textbf{Entry conditions} & ED is correctly logged in. The ED has decided to create a Tournament. The ED is on his profile page.        \\
        \hline
        \textbf{Event Flow}       & 1-\ The ED clicks on “Create a Tournament” button        \\
        & 2-\ CKB return the create Tournament form        \\
        & 3-\ The ED writes the Tournament name, sets the deadline for STs to subscribe and chooses and writes the other EDs nicknames to grant them permissions to create new Battles        \\
        & 4-\ CKB stores the data of the Tournament        \\
        & 5-\ CKB sends a notification to the chosen EDs        \\
        & 6-\ CKB notifies all the STs of the creation of the Tournament        \\
        & 7-\ CKB return the Tournament main page        \\
        \hline
        \textbf{Exit condition}   & The Tournament is correctly created and a confirmation message is shown to the ED        \\
        \hline
        \textbf{Exceptions}        & \begin{itemize}
            \item The Tournament’s name is null or already exists. An error message is shown and the ED is redirected back to create Tournament settings.
            \item The EDs chosen are non-existent, or the creator chooses other ED that have already permission in this Tournament. An error message is shown to the ED and the ED is redirected to create Tournament settings.
            \item The Deadline is a date before the day the Tournament is created. An error message is shown to the ED and a new Deadline has to be chosen and the ED  is redirected to create Tournament settings.
         \end{itemize}    \\
        \hline
        \caption{Create a Tournament use case}
        \label{tab: create_a_Tournament_use_case}
    \end{longtable}
\end{center}

\subsubsection*{UC\cuc . Join a Tournament}
\begin{center}
    \begin{longtable}{|l|p{0.9\linewidth}|}
        \hline
        \textbf{Actor}            & ST                                                                                                                                                                                        \\
        \hline
        \textbf{Entry conditions} & The Student is correctly logged in. The Student has decided to join a Tournament.                                                                                                                   \\
        \hline
        \textbf{Event Flow}       & 1-\ The Student search the name or the code of a Tournament into the search box \\
        & 2-\ CKB returns the result from the search. \\
        & 3-\ The Student clicks on the new Tournament’s name.        \\
        & 4-\ CKB redirects the Student to the main page of the Tournament. \\
        & 5-\ The Student clicks on the “Join Tournament” button. \\
        & 6-\ CKB saves the Student as one of the competitors of the Tournamen \\
        & 7-\ CKB shows a confirm message and redirects the Student to the page that contains the current Leaderboard of the Tournament. \\
        \hline
        \textbf{Exit condition}   & The Student is now part of the Tournament and will be notified whenever a new Battle will be available.        \\                                                                                                                                                                                \\
        \hline
        \textbf{Exceptions}        & \begin{itemize}
            \item The Tournament already ended. In this case is shown an error message, the User is redirected to the final leaderboard of the Tournament.
            \item The student already joined the Tournament. In this case CKB will not show any error messages, but will redirect the Student to the current Leaderboard of the Tournament.
            \item The Deadline already expired. An error message is shown and the ST is redirected back to the Home Page.
        \end{itemize}    \\
        \hline
        \caption{Join a Tournament use case}
        \label{tab: join_a_Tournament_use_case}
    \end{longtable}
\end{center}

\subsubsection*{UC\cuc . Create a Battle}
\begin{center}
    \begin{longtable}{|l|p{0.9\linewidth}|}
        \hline
        \textbf{Actor}            & ED                                                                                                                                                                                       \\
        \hline
        \textbf{Entry conditions} & ED is logged in. ED is in the Tournament page. ED has decided to create a Battle        \\
        \hline
        \textbf{Event Flow}       & 1-\ ED clicks on “Create new Battle” button        \\
        & 2-\ CKB redirect ED to the create Battle form        \\
        & 3-\ ED writes the Battle name        \\
        & 4-\ ED sets the minimum and maximum number of students per group        \\
        & 5-\ CBK checks the Battle name, the minimum and the maximum        \\
        & 6-\ CKB redirect ED to the upload code kata form        \\
        & 7-\ ED uploads the code kata to be resolved        \\
        & 8-\ CKB save the code kata sended        \\
        & 9-\ CKB redirect ED to the deadline and additional condition form        \\
        & 10-\ ED sets a registration deadline        \\
        & 11-\ ED sets a final submission deadline        \\
        & 12-\ ED sets additional configurations for scoring        \\
        & 13-\ CKB check deadline         \\
        & 14-\ CKB create a new Battle        \\
        & 15-\ CKB sends a notification to all STs subscribed to the Tournament         \\
        \hline
        \textbf{Exit condition}   & The Battle is correctly created and the ED is redirected to the Battle main page        \\
        \hline
        \textbf{Exceptions}        & \begin{itemize}
                \item The Battle’s name is null or already exists. An error message is shown, the ED is redirected back to the create Battle settings.
                \item The maximum number of students per group is lesser than the minimum.  An error message is shown to the ED and the ED is redirected to the create Battle settings.
                \item The minimum number of students per group is lesser or equal than zero.  An error message is shown to the ED and the ED is redirected to the create Battle settings.    
        \end{itemize}    \\
        \hline
        \caption{Create a Battle use case}
        \label{tab: create_a_Battle_use_case}
    \end{longtable}
\end{center}

\subsubsection*{UC\cuc . Join a Battle}
\begin{center}
    \begin{longtable}{|l|p{0.9\linewidth}|}
        \hline
        \textbf{Actor}            & ST                                                                                                                                                                                        \\
        \hline
        \textbf{Entry conditions} & The Student is correctly logged in and it is in the Tournament main page (the one with the leaderboard). The Student has decided to join a Battle.        \\
        \hline
        \textbf{Event Flow}       & 1-\ The Student click on the button “Go to Battles”        \\
        & 2-\ CKB redirects the student to the page that has the list of all the Battles of the Tournament.        \\
        & 3-\ The Student clicks on the “subscribe” button near the Battle name.        \\
        & 4-\ CKB redirects the Student to the CreateGroup page of the Battle.        \\
        & 5-\ CKB redirects the Student to the CreateGroup page of the Battle.        \\
        \hline
        \textbf{Exit condition}   & The Student, after the subscription deadline end, will be able to create a group and then confirm the participation to the Battle        \\
        \hline
        \textbf{Exceptions}        & \begin{itemize}
            \item The Student isn’t in the Tournament that this Battle is part of. In this case CKB shows an error message, the User is redirected back to his main page/dashboard
            \item The Battle already ended or already started. In this case  CKB shows an error message, the User is redirected to the final leaderboard of the Battle.
            \item The student already joined the Battle. In this case CKB will not show any error messages, but will redirect the Student to the CreateGroup page.
         \end{itemize}    \\
        \hline
        \caption{Join a Battle use case}
        \label{tab: join_a_Battle_use_case}
    \end{longtable}
\end{center}

\subsubsection*{UC\cuc . Create a group and Battle confirmation}
\begin{center}
    \begin{longtable}{|l|p{0.9\linewidth}|}
        \hline
        \textbf{Actor}            & ST                                                                                                                                                                                        \\
        \hline
        \textbf{Entry conditions} & The Student is correctly logged in and it is in the CreateGroup page. The Student has decided to join a Battle.        \\
        \hline
        \textbf{Event Flow}       & 1-\ The Student clicks on the “invite” button and then inserts another student’s nickname.        \\
        & 2-\ CBK checks if the student joined the competition and notifies other students of an invitation.          \\
        & 3-\ the other student can decide to accept or reject the invitation clicking on the “accept” or “reject” button inside the notification.        \\
        & 4-\ CKB shows to the Student that creates the group the different state of the various invitation requests (3 state : accepted, rejected, pending)(only the first n-invited players that accept the invite are considered part of the group)(where n is the upper bound threshold).         \\
        & 5-\ The Student can click on the “Join Battle” button.        \\
        & 6-\ CKB checks if the number of members respect the limits given by educators during the Battle creation.         \\
        & 7-\ CKB saves the Student and the group as one of the competitors of the Battle        \\
        & 8-\ CKB shows a confirmation message and redirects all the Students to the group to the main page of the Battle (the one with the leaderboard).        \\
        \hline
        \textbf{Exit condition}   & The Student is now a competitor in the Battle and he can starts committing after the creation group deadline expires        \\
        \hline
        \textbf{Exceptions}        & \begin{itemize}
            \item The group already joined the Battle. In this case CKB will not show any error messages, but will redirect the Student to the current Leaderboard of the Battle.
            \item The number of students in the group is wrong. CKB will show an error and redirect the students to the CreateGroup page.
            \item If the creation group deadline and all the students without group or all the groups not confirmed will be kicked out from the Battle
            \item if the invited student has not already joined in the competition or if the nickname does not exist, CKB sends an error and redirects the ST to the CreateGroup page.
         \end{itemize}    \\
        \hline
        \caption{Create a group and Battle confirmation use case}
        \label{tab: create_a_group_and_Battle_confirmation_use_case}
    \end{longtable}
\end{center}

\subsubsection*{UC\cuc . Open a profile}
\begin{center}
    \begin{longtable}{|l|p{0.9\linewidth}|}
        \hline
        \textbf{Actor}            & ST or ED                                                                                                                                                                                       \\
        \hline
        \textbf{Entry conditions} & The User is logged in. The User should be in a leaderboard (both Battle or Tournament) page for retreiving a ST profile from it or an ED profile from the owner name of the Tournament (or from the list of all the EDs)        \\
        \hline
        \textbf{Event Flow}       & 1-\ User clicks on a ST's nickname (from the leaderboard) or ED's nickname (from the Tournament owner or from the list of all the EDs)        \\
        & 2-\ CKB returns the profile page of the selected User        \\
        \hline
        \textbf{Exit condition}   & User is able to know particular information of the User that he had selected        \\
        \hline
        \textbf{Exceptions}        &  \\%TODO
        \hline
        \caption{Open a profile use case}
        \label{tab: open_a_profile_use_case}
    \end{longtable}
\end{center}

\subsubsection*{UC\cuc . Search for a profile}
\begin{center}
    \begin{longtable}{|l|p{0.9\linewidth}|}
        \hline
        \textbf{Actor}            & ST or ED                                                                                                                                                                                       \\
        \hline
        \textbf{Entry conditions} & The User is logged in  \\
        \hline
        \textbf{Event Flow}       & 1-\ User clicks on the search bar        \\
        & 2-\ User writes another User's nickname or a keyword of it \\
        & 3-\ CKB returns the list of the Users that contain the written keyword in their nickname         \\
        \hline
        \textbf{Exit condition}   & User is able to find the User that he was searching        \\
        \hline
        \textbf{Exceptions}        &  \\%TODO
        \hline
        \caption{Search for a profile use case}
        \label{tab: search_for_a_profile_use_case}
    \end{longtable}
\end{center}

\subsubsection*{UC\cuc . Search for a Tournament}
\begin{center}
    \begin{longtable}{|l|p{0.9\linewidth}|}
        \hline
        \textbf{Actor}            & ST or ED                                                                                                                                                                                       \\
        \hline
        \textbf{Entry conditions} & The User is logged in  \\
        \hline
        \textbf{Event Flow}       & 1-\ User clicks on the search bar        \\
        & 2-\ User writes a Tournament's name or a keyword of it \\
        & 3-\ CKB returns the list of the Tournaments that contain the written keyword in their name         \\
        \hline
        \textbf{Exit condition}   & User is able to find the Tournament that he was searching        \\
        \hline
        \textbf{Exceptions}        &  \\%TODO
        \hline
        \caption{Search for a Tournament use case}
        \label{tab: search_for_a_Tournament_use_case}
    \end{longtable}
\end{center}




\section{Performance requirements}
\label{sec:performance_requirements}%


\section{Design constraints}
\label{sec:design_constraints}%

\subsection{Standard compliance}
\label{subsec:standard compliance}%%
The system must be compliant to the EU's GDPR (General Data Protection Regulation), a set of regulations that is designed in order to protect the personal data, the privacy and security of the EU's citizens. 

\subsection{Hardware limitations}
\label{subsec:hardware_limitations}%
The only hardware limitations are the support for a reliable internet connection and for a Web Browser.

\subsection{Any other constraints}
\label{subsec:other_constraints}%
QESTA SEZIONE POTREMMO ANCHE TOGLIERLA DAL PDF


\section{Software system attributes}
\label{sec:software_system_attributes}%

\subsection{Reliability}
\label{subsec:reliability}%
The system has to be fault tolerant in order to prevent the propagation of errors and to guarantee a continuous usability of the system.

\subsection{Availability}
\label{subsec:availability}%
The system must be available the most time possible, with a minimum value of 99.9\% (three-nines) of time. In this way the system will be unavailable for only 8.76 hours a year. \\
It shall be prevented a case scenario in which a mainta break occurs near to Battle's end, therefore there must be as few maintenance breaks as possible, with them possibly at nightime.

\subsection{Security}
\label{subsec:security}%
The system must control the access rights of the users. The system shall grant both authentication, verifying the identity of the users that attempt to login and authorization, verifying the permission of the already logged users to perform certain requested actions. \\
Measures to protect the database will be adopted, such as defense against query injections, and password and users' personal data stored will be encrypted.


\subsection{Maintainability}
\label{subsec:maintainability}%
The system must be designed using scalable and reusable models in order to permit future addition of features with minimum effort. 
Ordinary maintenance has to be scheduled at nightime, in order to keep the services available when the user traffic is high.

\subsection{Portability}
\label{subsec:portability}%
The system must be accessible by the users from every kind of Web Browser. 
There are no particular portability requirements server side.



