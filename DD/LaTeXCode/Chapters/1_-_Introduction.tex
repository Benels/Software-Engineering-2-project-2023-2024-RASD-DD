\section{Purpose}
\label{sec:purpose}%

\section{Scope}
\label{sec:scope}%

\section{Definition, Acronyms, Abbreviations}
\label{sec:definition_acronyms_abbreviations}%

\begin{table}[H]
    \begin{center}
        \begin{tabular}{ |l|l| }
            \hline
            \textbf{Acronyms} & \textbf{Definition}                              \\
            \hline
            DD             & Design Document                      \\
            \hline
            ST              & Student                         \\
            \hline
            ED              & Educator                         \\
            \hline
            STG             & Student Group                    \\
            \hline
            CKB             & CodeKataBattle                   \\
            \hline
            GH              & GitHub                           \\
            \hline
            User            & All STs and EDs                           \\
            \hline
            API             & Application Programming Interface       \\
            \hline
            RX              & Requirement X                           \\
            \hline
            CMP            & Component                           \\
            \hline
         \end{tabular}
        \caption{Acronyms used in the document.}
        \label{tab:acronyms}%
    \end{center}
\end{table}

\section{Revision History}
\label{sec:revision_history}%

\section{Reference Documents}
\label{sec:reference_documents}%

\section{Document Structure}
\label{sec:doc_structure}%
The document is structured in seven sections, as described below.

In the first section, the chapter elucidates the significance of the Design 
Document, providing comprehensive definitions and explanations of acronyms and abbreviations. Additionally, it recalled the scope of the CodeKateBattle system.

The second section shows the main components of the system and their relationships. This section also focuses on design choices and architectural styles, patterns and paradigms.

The next section, the third, describes the user interface of the system, providing mockups and explanations of the main pages.

The fourth section describes the requirements of the system, showing how they are satisfied by the design choices.

This fifth part provides an overview of the implementation of the various components of the system, it also shows how they are integrated and it gives a plan for testing them all.

In the sixth section are included information about the number of hours each group member has worked for this document.

The last section contains the list of the documents used to redact this Design Document.
 
